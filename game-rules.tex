\section {Game Rules}
\label{game-rules}

\begin{enumerate}
\item The game, called \textbf{Caldera}, will be played in the arena defined in section~\ref{sub:arena}.  The objective of this game is to capture the most zones -- especially tricky, hard-to-reach ones -- and keep your opponents from doing the same.

\item Before a match begins, participating teams must:
\begin {enumerate}
  \item Present their robot in the staging area, adjacent to the arena, at least 2 minutes before the scheduled start time.
        The staging area will be clearly marked on the day.

  \item Attach a flag. The flag will be provided by Student Robotics officials in the staging area.

  \item Place their robot in the starting zone that they are assigned.
        The robot must be placed such that it is entirely within this starting zone, with no parts overhanging its boundary.

  \item Vacate the arena 40 seconds before the scheduled start time.
        During the 40 second period prior to the start of the match there must be no interaction with the robot.
\end{enumerate}
  Teams that fail to comply with this rule--such as by arriving late--may forfeit the match, at the discretion of the judge.

\item A match lasts 150 seconds.

\item There will be a maximum of 4 robots in a match.

\item Robots will be started by teams leaning into the arena to press the start button on their robot\footnote{A wireless match-starting solution may be provided by Student Robotics.} when instructed to do so.

\item A match may be terminated prematurely if all teams participating in that match state to the match officials that they are happy for the game to end.

\item There are 25 scoring zones in the arena, arranged in a $5\times5$ grid. The outer 16 are known as the ``base'', the inner 8 are known as the ``volcano'', and the most central zone is known as the ``caldera''. The volcano (and, notably, \emph{not} the caldera) is elevated from the floor by $50 mm \pm 5 mm$.

\item There are 40 tokens in the arena: 10 per team. These tokens are marked for the starting zone with which they are associated. These are initially placed in 5 stacks of 2 high, along the wall to the right of the robot's starting position, as defined in section~\ref{sub:arena}.

\item A scoring zone is deemed ``captured'' by a team if and only if that team has the single\footnote{Were two teams to have the same number of tokens in a scoring zone, the zone would not be considered captured by any team.} most tokens ``in'' the zone at the end of the game.

\item A token is considered to be `in' a zone if either:
\begin{enumerate}
  \item at least three vertices of the token are in contact with the floor area inside the zone. The floor area inside the zone is bounded by the edges of the raised volcano, and the inside edge of the coloured tape marking the zone.
  \item the token is in contact only with other tokens which are `in' the zone. In the case where several such tokens are in mutual contact (and not in contact with anything else), all those tokens are `in' the zone.
\end{enumerate}

\item Captured zones in the base are worth \textbf{2 game points} each. Captured zones on the volcano are worth \textbf{7 game points} each. The caldera, if captured, is worth \textbf{30 game~points}.

\item If, at the end of the game, a robot is ``in'' a zone, the value of that zone and the four zones immediately adjacent in the cardinal directions are tripled. This effect compounds with multiple robots, meaning that it is possible for the value of a zone to be increased 81-fold in one theoretical case.  % When a robot is 'in' a zone is explicitly undefined and will be made by a ruling at the competition

\item At the end of a game, league points will be awarded as follows.
      The team with the \emph{most} game points will be awarded 8 points towards the competition league.
      The team with the second most will be awarded 6.
      The team with the third most will be awarded 4 points, and the team with the fewest game points will be awarded 2 points.
      Teams whose robot was not entered into the round, or who were disqualified from the round, will be awarded no points.

      Tied robots will be awarded the average of the points that their combined positions would be awarded.
      Thus, three robots tied for first place would receive 6 points each (since this is $(8+6+4)/3$).

\item Once the league has completed, a knockout competition will begin.
      The positions of the teams in the league will seed the positions of teams in the knockout matches.
      The top teams from the league advance to the knockout.
      The number of teams progressing to the knockout will be announced before the start of the league matches.
      In the event of tied league positions, the team with the greatest cumulative game points in the league will go through.

      Each match in the knockout competition involves up to 4 teams.
      The teams that come 1\textsuperscript{st} and 2\textsuperscript{nd} in each knockout match will continue to the next round of the knockout.
      In the event of a tie in a knockout match, the team that ranked highest in the league will go through.
      If there is a tie in the final, then a rematch will be played.
      The number of league and knockout matches will be announced on the morning of the competition.

\end{enumerate}

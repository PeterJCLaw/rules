\section {Game Rules}
\label{game-rules}

\begin{enumerate}
\item The game, called \textbf{Two Colours}, will be played in the arena defined in section~\ref{sub:arena}.  The objective of this game is to capture the most tokens, but without mixing the two colours.

\item Before a match begins, participating teams must:
\begin {enumerate}
  \item Present their robot in the staging area, adjacent to the arena, before the scheduled close of staging time.
        The staging area will be clearly marked on the day.

  \item Attach a robot flag.
        Robot flags will be provided by Student Robotics officials in the staging area.
        Section~\ref{sub:robot-flag} provides more information about these flags, as well as their dimensions and mounting requirements.

  \item Follow the directions of the match officials.
\end{enumerate}
  Teams that fail to comply with these rules--such as by arriving late--may forfeit the match, at the discretion of the judge.

\item A match lasts 150 seconds.

\item There will be a maximum of 4 robots in a match.

\item Robots will be started by, or at the direction of, match officials.

\item A match may be terminated prematurely if all teams participating in that match state to the match officials that they are happy for the game to end.

\item There are 16 tokens in the arena; 8 of them are gold, 8 silver. 4 of each colour are on the floor of the arena, and 4 are on a raised platform in the centre of the arena.

\item Each token, regardless of colour, is worth \textbf{3} points.

\item Points are awarded at the end of the game.

\item A token is ``in'' a scoring zone if, and only if, any part of it is in contact with the floor in the zone.

\item If tokens of both colours are present in a scoring zone, the value of each token in that scoring zone is reduced from \textbf{3} to \textbf{1}.

\item There is a bonus point available for leaving the scoring zone for the first time in a game.

\item At the end of a game, league points will be awarded as follows.
      The team with the \emph{most} game points will be awarded 8 points towards the competition league.
      The team with the second most will be awarded 6.
      The team with the third most will be awarded 4 points, and the team with the fewest game points will be awarded 2 points.
      Teams whose robot was not entered into the round, or who were disqualified from the round, will be awarded no points.

      Tied robots will be awarded the average of the points that their combined positions would be awarded.
      Thus, three robots tied for first place would receive 6 points each (since this is $(8+6+4)/3$).

\item Once the league has completed, a knockout competition will begin.
      The positions of the teams in the league will seed the positions of teams in the knockout matches.
      The top teams from the league advance to the knockout.
      The number of teams progressing to the knockout will be announced before the start of the league matches.
      In the event of tied league positions, the team with the greatest cumulative game points in the league will go through.

      Each match in the knockout competition involves up to 4 teams.
      The teams that come 1\textsuperscript{st} and 2\textsuperscript{nd} in each knockout match will continue to the next round of the knockout.
      In the event of a tie in a knockout match, the team that ranked highest in the league will go through.
      If there is a tie in the final, then a rematch will be played.
      The number of league and knockout matches will be announced on the morning of the competition.

\end{enumerate}

\section{Scoring}
\label{sec:scoring}

The score of any particular configuration of captured squares is given
by taking the permutation of captured squares which gives the maximum
value for the following procedure.

Given the ordered list $L$ of captured squares, the score is defined
inductively:

\begin{itemize}
  \item If $L$ is the empty list, the score is $0$.

  \item If $L$ is a non-empty list, we define $H$ as its first element, and
  $T$ as its `tail', that is, every element from the second onwards, if any.

  \item If $T$ contains two squares in the same row as $H$, the score is $6$
  plus the score for the list $T$.

  \item Otherwise, if $T$ contains two squares in the same column as $H$,
  the score is $6$ plus the score for the list $T$.

  \item Otherwise, if $T$ contains one square in the same row as $H$, the
  score is $3$ plus the score for the list $T$.

  \item Otherwise, if $T$ contains one square in the same column as $H$,
  the score is $3$ plus the score for the list $T$.

  \item Otherwise, the score is $1$ plus the score for the list $T$.
\end{itemize}

In addition to the above procedure, $1$ point is added for initial movement
of the robot.

This scoring system has the following properties:
\begin{itemize}
  \item Invariance under rotation by 90 degrees.
  \item Invariance under reflection in either diagonal axis.
  \item Invariance under permutation of rows and columns.
\end{itemize}


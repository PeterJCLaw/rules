\section {Game Rules}
\label{game-rules}

\begin{enumerate}
\item The game, called \textbf{Pirate Plunder}, will be played in the arena defined in section~\ref{sub:arena}.  The objective of this game is to achieve as many points as possible by collecting tokens, placing them in buckets, and moving buckets and tokens into the team's zone.

\item Before a match starts, the teams participating in that match will be given some time to set their robot up in the arena.  During this time, they:
\begin{enumerate}
  \item Must place their robot in the zone that they are assigned.  The robot must be placed such that it is entirely within this zone, with no parts overhanging its boundary.

  \item May position the bucket associated with their zone to be in any legal location within their zone.  The bucket must be placed such that it is entirely within the allocated zone.

  \item Must ensure their robot has four ``robot markers'' attached it.  These will be provided by Student Robotics officials at the beginning of the set-up time.  Section TODO provides more information about these markers, as well as their dimensions and mounting requirements.
\end{enumerate}

\item At the end of a match, each team's ``\textbf{game points}'' will be calculated.
 These are used to rank teams before competition league points are awarded.  Game points will be awarded as follows:
\begin{itemize}
  \item \textbf{1 point} will be awarded for each token that the robot is carrying.
  \item For each token that is entirely within, and in contact with the floor of the team's zone, \textbf{2 points} will be awarded.
  \item \textbf{5 points} will be awarded for each token that is inside a bucket that is entirely within the team's zone.

  \item The \textbf{total score} of a team will be multiplied by the number of buckets that are entirely within their zone.

\end{itemize}

\item A robot will be considered to be carrying a token if the token's weight is fully supported by the robot, and the token is not in contact with any part of the arena (walls, floor, etc.).

\item At the end of a game, the team with the \emph{most} game points will be awarded 4 points towards the competition league.
 The team with the second most will be awarded 3.
 The team with the third most will be awarded 2 points, and the team with the fewest game points will be awarded 1 point.
 Teams whose robot was not entered into the round, or who were disqualified from the round, will be awarded no points.

\item There will be a maximum of 4 robots in a match.
\item A match lasts 180 seconds.
\item Robots will be started by teams leaning into the arena to press the start button on their robot\footnote{A wireless match-starting solution may be provided by Student Robotics} when instructed to do so.

\item Teams that do not present their robot promptly for a match will forfeit that match.
\end{enumerate}

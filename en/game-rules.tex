\section {Game Rules}
\label{game-rules}

\begin{enumerate}
\item The game, called \textbf{A Strange Game}, will be played in the arena defined in section~\ref{sub:arena}.
      The objective of this game is to achieve as many points as possible by placing team associated tokens in squares and pedestals to capture them.

\item Before a match starts, the teams participating in that match will be given some time to set their robot up in the arena.
      During this time, they:
\begin{enumerate}
  \item Must place their robot in the starting zone that they are assigned.
        The robot must be placed such that it is entirely within this starting zone, with no parts overhanging its boundary.

  \item Must ensure their robot has four robot badges attached it.
        These will be provided by Student Robotics officials at the beginning of the set-up time.
        Section~\ref{sec:robot-badges} provides more information about these badges, as well as their dimensions and mounting requirements.
\end{enumerate}
      Once all robots been arranged, 6 tokens per robot will be placed adjacent to their respective starting zones.

\item A match lasts 180 seconds.

\item There will be a maximum of 4 robots in a match.

\item At the end of a match, each team's ``\textbf{game points}'' will be calculated.
      These are used to rank teams before competition league points are awarded.
      Game points will be awarded as follows:
\begin{itemize}
  \item \textbf{1 point} will be awarded for initial movement outside the starting zone, defined as when the back of the robot passes over the boundary.
  \item For each row of squares, \textbf{1 point} will be awarded if one square is captured, \textbf{3 points} will be awarded if two squares are captured, and \textbf{6 points} will be awarded if all squares are captured.
  \item For each column of squares, \textbf{1 point} will be awarded if one square is captured, \textbf{3 points} will be awarded if two squares are captured, and \textbf{6 points} will be awarded if all squares are captured.
  \item For each collision with another robot where clearly at fault, \textbf{1 point} will be deducted.
  \item At the end of the round, points will be clamped to zero if negative.
\end{itemize}

\item Ownership of a square will be determined as follows:
\begin{enumerate}
  \item If there are one or more tokens on the central pedestal, the square will be deemed to have been captured by the robot associated with the highest token in the stack.
  \item Otherwise, if one robot has more tokens in the square than any other (excluding those which might happen to fall across square boundaries, which count towards neither square), that robot is deemed to have captured the square.
  \item Otherwise, the square is deemed unclaimed.
\end{enumerate}

\item At the end of a game, league points will be awarded as follows.
      The team with the \emph{most} game points will be awarded 4 points towards the competition league.
      The team with the second most will be awarded 3.
      The team with the third most will be awarded 2 points, and the team with the fewest game points will be awarded 1 point.
      Teams whose robot was not entered into the round, or who were disqualified from the round, will be awarded no points.

      Tied robots will be awarded the average of the points that their combined positions would be awarded.
      Thus, three robots tied for first place would receive 3 points each (since this is $(4+3+2)/3$).

\item Once the league has completed, a knockout competition will begin.
      The positions of the teams in the league will seed the positions of teams in the knockout matches.
      The top 24 teams from the league advance to the knockout.
      In the event of tied league positions, the team with the greatest cumulative game points in the league will go through.

      Each match in the knockout competition involves up to 4 teams.
      The teams that come 1\textsuperscript{st} and 2\textsuperscript{nd} in each knockout match will continue to the next round of the knockout.
      In the event of a tie in a knockout match, the team that ranked highest in the league will go through.
      If there is a tie in the final, then a rematch will be played.
      The number of league and knockout matches will be announced on the morning of the competition.

\item Robots will be started by teams leaning into the arena to press the start button on their robot\footnote{A wireless match-starting solution may be provided by Student Robotics.} when instructed to do so.

\item There must be no team members in the arena during the 1 minute before a match is scheduled to start.
      Robots must be installed and oriented before this deadline.
      During this minute there must be no interaction with the robot.
      Teams that do not meet this rule will forfeit the match.

\item A match may be terminated prematurely if all teams participating in that match state to the judge that they are happy for the game to end.

\item A token will be considered to be on a pedestal if the token is fully supported by the pedestal, and no part of the token is in contact with a robot, or any other part of the arena.

\end{enumerate}

\section {Règlements du Jeu}
\label{game-rules}

\begin{enumerate}
\item Le jeu, appelé \textbf{Tin-Can Rally}, (Rallye Boîte de Conserve) sera joué dans l'arène défini en\autoref{sub:arena}.
\item Avant le début du jeu, deux boîtes seront placées au hasard dans les quatre couloirs qui se trouvent a chaque côté de l'arène. La «superboîte» sera placée au sommet de la rampe.
\item Des équipes se seront attribuées un coin de l'arène que où leur robot va commencer au début du jeu.
Le robot doit être placé au sein de $100mm$ des deux murs de l'arène.
\item À la fin d'un match, les ``\textbf{points du jeu}'' de chaque équipe participante seront calculé.
 Ces points seront utilisés pour classer des équipes afin de décerner des points de la ligue.

\item Les points du jeu seront décernés comme suit:

\begin{itemize}
\item Quand un robot se déplace de sa position de départ, \textbf{1 point} sera décerné.
\item Lorsque le dos d'un robot passe la frontière d'un quadrant, comme définie dans \autoref{sub:arena}, \textbf{2 points} seront décernés.
\item Lorsqu'un robot porte une boîte (y compris la «superboîte») par-dessus de la frontière d'un quadrant, \textbf{1 point} sera décerné.
\item Lorsqu'un robot termine son ascension de la rampe, défini par la passage de son dos par-dessus la fin du plateau de la rampe, \textbf{3 points} seront décernés.
\item Lorsqu'un robot termine son descente de la rampe, défini par la passage de son dos par-dessus la fin de la pente de la rampe, \textbf{3 points} seront décernés.
\item Quand un robot termine une jeu, supportant \footnote{Pour savoir si un robot supporte une boîte, il sera levé hors de toute surface de l'arène qui peut supporter une boîte.} une boîte normale \textbf{2 points} seront décernés.
\item Quand un robot termine une jeu, supportant la «superboîte», leur \textbf{score total} pour ce jeu sera \textbf{doublé}.
\end{itemize}

\item Un robot sera consideré de porter une boîte si la boîte estA robot will be considered to be carrying a token if the token is both:
  \begin{enumerate}
  \item poussée par le robot
  \item en contact avec le robot, mais pas les murs ou le sol de l'arène.
  \end{enumerate}

\item Points peuvent seulement être cumulés par des robots qui voyagent dans le sens anti-horaire par rapport au centre de l'arène

\item À la fin d'un jeu, l'équipe ayant \emph{le plus} des points de jeu sera décerné 4 points vers le championnat du concours.
 L'équipe en deuxième place sera décerné 3 points vers le championnat.
 L'équipe en troisième place sera décerné 2 points, et l'équipe avec le moins des points du jeu sera décerné 1 point.
 Les équipes dont le robot n'a pas été entré dans le match, ou qui ont été exclus du match, seront décernées aucun points.

\item Il y aura un maximum de 4 robots par match.
\item Un match durera 180 seconds.
\item Les matchs seront démarrés et arrêtés par le sytème Infrarouge de Student Robotics\footnote{Le sytème Infrarouge de Student Robotics sera relié aux robots avant qu'ils entrent dans l'arène pour leur match. Il sera utilisé pour la sécurité de coupure, et les signaux démarre-match et arrête-match.}.
\item Des équipes qui ne présentent pas leur robot rapidement pour un match, perdront ce match.
\end{enumerate}

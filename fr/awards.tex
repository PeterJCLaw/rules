\section {Prix}
\label{sec:Awards}

\subsection{Les Prix Principaux}
Ces prix seront remis aux équipes qui à la fin du concours. Les équipes en $1^{ere}$, $2^{eme}$ et $3^{eme}$ place recevront des prix.

\subsection{Prix du Président}
Le Prix du Président sera décerné à l'équipe qui montre le plus extraordinaire ingéniosité dans la conception de leur robot. Il ne sera pas attribué à la complexité de la conception, plutôt la mise en œuvre d'une solution simple et élégante du jeu-concours.

\subsection{Premiere mouvement du Robot}
L'équipe qui démontre le premier robot qui est mobile à la communauté sera décerné un prix mangeable au concours.
\begin{enumerate}
\item Le mouvement du robot doit être contrôlé par un logiciel installé sur le kit électronique de Student Robotics
\item Le robot doit se déplacer 1 mètre, et puis s'arrêter, sans intervention.
\item La preuve sera obtenu en téléchargeant une vidéo du robot sur un site public de vidéos en ligne (e.g. youtube.com, flickr.com).
\end{enumerate}


\subsection{Présence en ligne}
L'équipe qui est jugé d'avoir la meilleure présence en ligne sera décerné un prix mangeable au concours. Une présence en ligne est un ensemble public des pages qui décrive les progrès de l'équipe, et peut se composer d'un blog, des photos et des vidéos de l'équipe et du robot.  \emph{Astuce: Sites utiles incluent blogger.com, wordpress.com, flickr.com and youtube.com}
\begin{enumerate}
\item Lorsque vous décrivez vos activités, ne publiez pas des informations privées concernant vous ou des autres.
\item Une fois que c'est disponible, dites-nous où nous pouvons trouver vos publications, à \href{mailto:info@studentrobotics.org}{\nolinkurl{info@studentrobotics.org}}
\end{enumerate}

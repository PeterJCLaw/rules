\section {Règlements Générales}
\label{sec:Regulations}

\begin{enumerate}
\item Un robot doit passer une inspection par un inspecteur de Student Robotics avant de participer dans un match.
 \textbf{Robots qui ne passent pas cette inspection ne seront pas autorisé à concourir}.
\item Au début de chaque match, les robots doivent etre capable de se placer dedans un cube de $500mm$ côtés internes.
\item La carte de puissance, y compris son bouton marche-arret, doit être facilement accessible.
 C'est pour votre sécurité, et la sécurité des autres autour de vous
\item Tous les composants électroniques personnalisées qui nécessitent une connexion à la batterie  doit plutôt être reliée au connecteur d'alimentation pour la carte moteur.
 Il y a un connecteur supplémentaire sur la carte de puissance à cet effet.
\item Tous les fils reliés à la masse du robot (la ligne 0V) doivent être noir.
 Fils noirs \emph{ne peuvent pas} être utilisées pour rien d'autre.
 Il est \emph{fortement recommandé} que tout le câblage est soignée et facilement démontable, car cela réduirait le temps nécessaire pour déboguer des problèmes sur les robots 
 (équipes peuvent être invités à ranger leur câblage avant qu'un Ingénieur de Student Robotics approche aucun problème avec leur robot).
\item Tous les composants électroniques doivent être solidement fixées au robot, et devrait également être facilement amovible.
\item Aucun système de télécommande peut être utilisé, à l'exception du Système Infrarouge de Student Robotics, pour démarrer et arrêter des matchs.
\item Ceci est un sport sans contact, mais les bosses accidentelles et les égratignures sont inévitables.
\item Il ne doit pas être possible de se blesser sur le robot.
 Ce sera testée en utilisant une saucisse de Francfort pour simuler un doigt.
\item Les robots doivent pas endommager les boîtes, la rampe, l'arène ou d'autres robots intentionnellement.
\item Les robots doivent avoir un mât fixé. Le mât est la seule partie du robot qui peut dépasser la limite de hauteur de $500mm$  Voir \autoref{sub:Flags} pour plus de détails.
\item Les robots doivent être entièrement verte le long de leurs côtés et la surface dessus, à l'exception de la webcam. 
 C'est pour aider le système de vision.
\item Si une équipe veut ajouter des systèmes alimentés par des autres battéries, ils doivent obtenir l'approbation de Student Robotics avant de commencer.
 En général, les équipes sont encouragées à utiliser la batterie qui a été fourni par Student Robotics, en utilisant les contacteurs sur la carte de puissance.
 Tous les composants électromécaniques \emph{doivent} être alimentés par connecteur d'alimentation pour la carte moteur fourni sur la carte de puissance.
\item Nous réservons le droit de regarder à votre logiciel, et votre matériel à tout moment.
\item Assistance d'Ingénieurs de Student Robotics est fourni sans aucune garantie.
\item Tous les kits fournis par Student Robotics reste la propriété de Student Robotics.
 Tous les kits électroniques \emph{doivent} être rendus à Student Robotics après le concours. Voir \autoref{sec:kit-return} pour plus de détails.
\item La décision du juge est définitive.


\end{enumerate}

\section {Awards}
\label{sec:Awards}

\subsection{Main Competition Awards}
Prizes will be awarded to the teams that are placed highest at the end of the competition.
The teams in $1^{st}$, $2^{nd}$ and $3^{rd}$ place will receive awards.

\subsection{Rookie Award}
The Rookie Award will be awarded to the rookie
team\footnote{A rookie team is one from a school, college or independent group that hasn't competed in a Student Robotics competition before, nor certain similar competitions: Robocon 2018 and SourceBots 2018.}\addtocounter{footnote}{-1}\addtocounter{Hfootnote}{-1}
 that places highest in the league.

\subsection{Committee Award}
The Committee Award will be given to the team that displays the most extraordinary ingenuity in the design of their robot.
It will not be awarded for complexity of design, rather the implementation of a simple and elegant solution to a problem.

\subsection{Robot and Team Image}
The team that presents their robot and themselves in what is judged to be the most outstanding way will receive this award.

\subsection{First Robot Movement}
The first rookie team\footnotemark{} that demonstrates a moving robot to the community will be awarded with an edible prize at the final competition.
\begin{enumerate}
\item The robot movement must be controlled by software running on the Student Robotics kit.
\item The robot must move 2 metres, pause for 2 seconds, turn $180\degree$ ($\pm20\degree$), return to its starting area ($\pm0.5m$), and come to a halt without interference.
\item This must be demonstrated by a video on the web (e.g. on YouTube, flickr, etc.) and linking to this video from a post on the Student Robotics forum.
\end{enumerate}

\subsection{Online Presence}
The team that is judged to have the best online presence will be awarded with an edible prize at the final competition.
\begin{enumerate}
\item The hashtag for this competition is \texttt{\#srobo\StrRight{\sryear{}}{3}}.
\item Teams are encouraged to post their activity and online presence in the Student Robotics forums.
\item When detailing activities online do not post any private information concerning yourself or others.
\end{enumerate}

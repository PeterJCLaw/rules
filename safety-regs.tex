\section {Safety regulations}
\label{sec:safety-regs}

To maintain safety at the competition, all robots at the event are required to
pass the safety regulations that are listed below. Robots that do not comply
to these rules will not be permitted to compete.

These regulations are intended to identify a base level of safety --- the
inspector will use their own judgement and common sense when assessing
your robot, and your robot may be judged to be unsafe for reasons or features
not listed here.

We recommend that you bear these regulations in mind during development too,
although it's not always possible to meet them while building and testing
your robot.

The following procedure will be used when testing a robot:

\begin{enumerate}
  \item Check that the parts of the robot that were provided by Student Robotics
  are in a safe condition. If any of the following criteria are not met, the
  offending component must be replaced with one in suitable condition, and the
  procedure restarted from the beginning.
  \begin{enumerate}
    \item Check that the cables between the power board and body of the battery
    are not damaged. The sheath must not have any holes in it.
    \item Locate the yellow XT60 connector pair that joins the battery to the
    cable leading back to the power board. Check that the insulation surrounding
    these connectors and the attached wiring is undamaged.
    \item In turn, give each of the wires attached to these connectors a gentle
    tug. The cables must not move relative to the connectors.
    \item Locate the metal terminals that connect the battery cable to the power
    board. Check that the insulation surrounding these terminals and the
    attached wiring is undamaged.
    \item In turn, give each of the wires attached to these terminals a gentle
    tug. The cables must not move.
  \end{enumerate}
  \item Check that the parts of the robot that were built by the competitors are
  in a safe condition. If any of the following criteria are not met, the team
  must be instructed to make amendments to the robot.
  \begin{enumerate}
    \item Check that there is a battery installed in the robot.
    \item Check that any additional power sources have already been authorised.
    \item Leaving the battery physically installed, unplug the XT60 connector.
    \item Check the battery's mounting holds the battery securely, and does not
    expose the battery to sharp edges.
    \item Check that the battery's casing is rigid, and strong -- i.e. bubble
    wrap is not suitable.
    \item Check that the cables between the power board and body of the battery
    do not pass through areas of the robot that could cause them to be damaged
    by moving mechanical parts.
    \item Check that only the power board is connected to the battery (if the
    XT60 connector were currently connected).
    \item Check that the power switch on the power board is easily accessible.
    \item Check that all electronics are securely fixed to the robot.
    \item Check for unreasonably sharp edges and dangerous moving parts.
  \end{enumerate}
\end{enumerate}

\section {Safety regulations}
\label{sec:safety-regs}

To maintain safety at the competition, all robots at the event are required to
pass the safety regulations that are listed below. Robots that do not comply
to these rules will not be permitted to compete.

These regulations are intended to identify a base level of safety --- the
inspector will use their own judgement and common sense when assessing
your robot, and your robot may be judged to be unsafe for reasons or features
not listed here.

We recommend that you bear these regulations in mind during development too,
although it's not always possible to meet them while building and testing
your robot.

\subsection{Regulations}

The following procedure will be used when testing a robot:

\begin{itemize}
\item Check that there is a battery installed in the robot.
\item Check that any additional power sources have already been authorised.
\item Check that the battery cable originally provided by Student Robotics is being used to connect the power board to the battery. If not, check that the replacement has suitable rating and quality.
\item Leaving the battery physically installed, unplug the XT60 connector.
\item Check the battery's mounting holds the battery securely, and does not expose the battery to sharp edges.
\item Check that the battery's casing is rigid, and strong -- i.e. bubble wrap is not suitable.
\item Locate the metal terminals that connect the battery cable to the power board. In turn, give each of the wires attached to these terminals a gentle tug. The cables must not move.
\item Check that the cables between the power board and body of the battery are not damaged. The sheath must not have any holes in etc.
\item Check that the cables between the power board and body of the battery do not pass through areas of the robot that could cause them to be damaged by moving mechanical parts.
\item Check that only the power board is connected to the battery (if the XT60 connector were currently connected).
\item Check that the power switch on the power board is easily accessible.
\item Check that all electronics are securely fixed to the robot.
\item Check for unreasonably sharp edges and dangerous moving parts.
\end{itemize}

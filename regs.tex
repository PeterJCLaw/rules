\section {Regulations}
\label{sec:Regulations}

\begin{enumerate}

% Overarching safety

\item All robots must be safe.

\begin{enumerate}
  \item It must not be possible to directly or indirectly injure oneself on the robot.
        Exposed sharp edges and fast moving parts, for example, will be tested using a Frankfurter sausage to simulate a finger.
        Teams are encouraged to discuss any safety concerns about their robot on the Student Robotics forums.

  \item The robot's power switch must be on the outside top of the robot and easily accessible at all times -- including throughout the game.
        This is for everyone's safety, especially your robot's.

  \item The lithium-ion polymer batteries provided in the kit must be shielded from mechanical and thermal harm.
        This includes mechanical protection from accidental impact with other robots.
        Teams found to be in violation of this rule will have their batteries confiscated until they have demonstrably rectified the identified issues.

  \item Only the power board may be connected directly to the battery.
\end{enumerate}

%% Meta

\item The Judge's decision is final.
\item Any assistance from Student Robotics volunteers is provided without guarantees.
\item Student Robotics reserves the right to examine your robot software and hardware at any time.

%% Behaviour

\item No remote control systems may be used.
\item This is a non-contact sport, but accidental bumps and scrapes are inevitable.
\item Robots must not intentionally damage anything -- including tokens, zone barriers, the arena or other robots.
      At the discretion of the judge, teams who deliberately engage in collisions or take insufficient precautions against collisions may be penalised, including disqualification from rounds and deduction of league points.
\item All kit deployed by Student Robotics remains the property of Student Robotics.
      All electronic kit \textbf{must} be returned to Student Robotics after the competition.
      \autoref{sec:kit-return} details the parts of the kit that must be returned.
      After the competition, the kit that is not specified in \autoref{sec:kit-return} becomes the property of the team.

%% Physical

\item Robots must pass an inspection by a Student Robotics Inspector before competing in a match.
      This inspector will check that the robot complies with the rules and regulations of this game, and is safe to compete (see \autoref{sec:safety-regs}).
      \textbf{Robots that have not passed inspection will not be permitted to compete}.

\item At the beginning of each match, robots must fit within a cube with $500mm$ internal sides.
      \textit{During the match}, the robot may extend beyond this size.

\item During a match robots may deploy supporting equipment into the arena, as long as that equipment is clearly designed to be of direct benefit to the robot.
      Such equipment must not deliberately impede other robots and reasonable care must be taken to ensure that it does not become merely an obstacle to other robots.

\item All wires connected to the robot's ground (0V line) must be black.
      Black wires \emph{must not} be used for anything else.
      It is \emph{strongly recommended} that all wiring is neat and easily removable, as this will reduce the time required to debug problems on robots
       (teams may be asked to tidy their wiring before a Student Robotics volunteer will approach any issues with their robot).

\item All electronics must be securely fixed to the robot, and should also be easily removable.

\item All robots must have mountings for the removable robot flags
      provided by Student Robotics, as described in section~\ref{sub:robot-flag}. A mounting
      must firmly hold a flag in an upright position. Flags must be mounted on the top of the robot.

\item If teams wish to use batteries other than the lithium-ion polymer batteries provided,
       then they must seek approval from Student Robotics through the Student Robotics forums first.
      Additionally, if teams wish to add systems powered by separate batteries then they must seek approval through the same channel first, with the exception of video cameras.

      In general, teams are encouraged to power everything off the Student Robotics supplied battery through the power board.

\item Robots may not include radio transmitters or receivers.
      In exceptional circumstances, teams may request an exemption from this rule.

\item Robots must not have any devices designed for the sole purpose of producing audible noise, with the exception of the piezoelectric buzzer on the power board.

\end{enumerate}

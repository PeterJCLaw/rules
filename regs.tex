\section {Regulations}
\label{sec:Regulations}

\begin{enumerate}
\item Robots must pass an inspection by a Student Robotics Inspector before competing in a match.
 \textbf{Robots that have not passed inspection will not be permitted to compete}.
\item At the beginning of each match, robots must fit within a cube with $500mm$ internal sides.
\item The power board, including its power switch, must be easily accessible.
 This is for your safety, and the safety of others around you.
\item All custom electronics that require a connection to the battery must instead be connected to the motor rail.
 There is an extra connector on the power board for this purpose.
\item All wires connected to the robot's ground (0V line) must be black.
 Black wires \emph{must not} be used for anything else.
 It is \emph{strongly recommended} that all wiring is neat and easily removable, as this will reduce the time required to debug problems on robots
  (teams may be asked to tidy their wiring before a Student Robotics Engineer will approach any issues on their robot).
\item All electronics must be securely fixed to the robot, and should also be easily removable.
\item No remote control systems may be used, with the exception of the Student Robotics Infrared system for starting and stopping matches.
\item This is a non-contact sport, but accidental bumps and scrapes are inevitable.
\item It must not be possible to injure oneself on the robot.
 This will be tested using a Frankfurter sausage to simulate a finger.
\item Robots must not intentionally damage tokens, the ramp, the arena or other robots.
\item Robots may have a flagpole attached.
 The flagpole must be vertically mounted on the robot and $1m$ tall (when mounted on the robot and measured from the ground).
 The flagpole is the only part of the robot that may exceed the $500mm$ height limit.
 See \nameref{sub:Flags} for more details.
\item Robots must be entirely green along their sides and top surface, with the exception of the camera.
 This is to aid the vision system.
\item If teams wish to add systems powered by separate batteries then they must seek approval from Student Robotics first.
 In general, teams are encouraged to power everything off the SR-supplied battery through the power board.
 All electromechanical components \textbf{must} be powered through the motor rail provided by the power board.
\item Student Robotics reserves the right to look at your robot software and hardware at any time.
\item Assistance from Student Robotics Engineers is provided without any guarantees.
\item All kit deployed by Student Robotics remains the property of Student Robotics.
 All electronic kit \textbf{must} be returned to Student Robotics after the competition.
\item The Judge's decision is final.


\end{enumerate}
